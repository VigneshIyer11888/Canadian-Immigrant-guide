% Options for packages loaded elsewhere
\PassOptionsToPackage{unicode}{hyperref}
\PassOptionsToPackage{hyphens}{url}
%
\documentclass[
]{article}
\usepackage{lmodern}
\usepackage{amssymb,amsmath}
\usepackage{ifxetex,ifluatex}
\ifnum 0\ifxetex 1\fi\ifluatex 1\fi=0 % if pdftex
  \usepackage[T1]{fontenc}
  \usepackage[utf8]{inputenc}
  \usepackage{textcomp} % provide euro and other symbols
\else % if luatex or xetex
  \usepackage{unicode-math}
  \defaultfontfeatures{Scale=MatchLowercase}
  \defaultfontfeatures[\rmfamily]{Ligatures=TeX,Scale=1}
\fi
% Use upquote if available, for straight quotes in verbatim environments
\IfFileExists{upquote.sty}{\usepackage{upquote}}{}
\IfFileExists{microtype.sty}{% use microtype if available
  \usepackage[]{microtype}
  \UseMicrotypeSet[protrusion]{basicmath} % disable protrusion for tt fonts
}{}
\makeatletter
\@ifundefined{KOMAClassName}{% if non-KOMA class
  \IfFileExists{parskip.sty}{%
    \usepackage{parskip}
  }{% else
    \setlength{\parindent}{0pt}
    \setlength{\parskip}{6pt plus 2pt minus 1pt}}
}{% if KOMA class
  \KOMAoptions{parskip=half}}
\makeatother
\usepackage{xcolor}
\IfFileExists{xurl.sty}{\usepackage{xurl}}{} % add URL line breaks if available
\IfFileExists{bookmark.sty}{\usepackage{bookmark}}{\usepackage{hyperref}}
\hypersetup{
  pdftitle={Comprehensive Canadian Immigrant Guide},
  pdfauthor={Vignesh C Iyer},
  hidelinks,
  pdfcreator={LaTeX via pandoc}}
\urlstyle{same} % disable monospaced font for URLs
\usepackage[margin=1in]{geometry}
\usepackage{graphicx,grffile}
\makeatletter
\def\maxwidth{\ifdim\Gin@nat@width>\linewidth\linewidth\else\Gin@nat@width\fi}
\def\maxheight{\ifdim\Gin@nat@height>\textheight\textheight\else\Gin@nat@height\fi}
\makeatother
% Scale images if necessary, so that they will not overflow the page
% margins by default, and it is still possible to overwrite the defaults
% using explicit options in \includegraphics[width, height, ...]{}
\setkeys{Gin}{width=\maxwidth,height=\maxheight,keepaspectratio}
% Set default figure placement to htbp
\makeatletter
\def\fps@figure{htbp}
\makeatother
\setlength{\emergencystretch}{3em} % prevent overfull lines
\providecommand{\tightlist}{%
  \setlength{\itemsep}{0pt}\setlength{\parskip}{0pt}}
\setcounter{secnumdepth}{-\maxdimen} % remove section numbering

\title{Comprehensive Canadian Immigrant Guide}
\author{Vignesh C Iyer}
\date{7/27/2020}

\begin{document}
\maketitle

\begin{center}\rule{0.5\linewidth}{0.5pt}\end{center}

\hypertarget{first-things-to-do-after-landing-in-canada}{%
\section{First Things to Do After Landing in
Canada}\label{first-things-to-do-after-landing-in-canada}}

\begin{center}\rule{0.5\linewidth}{0.5pt}\end{center}

\hypertarget{when-you-arrive-in-canada-you-must-have}{%
\subsubsection{When you arrive in Canada You must
have:}\label{when-you-arrive-in-canada-you-must-have}}

\begin{itemize}
\item
  your valid passport and/or travel documents

  \begin{itemize}
  \item
    your passport must be a regular, private citizen passport
  \item
    you can't immigrate to Canada with a diplomatic, government service
    or public affairs passport
  \end{itemize}
\item
  your Confirmation of Permanent Residence (COPR) and your permanent
  resident visa
\item
  proof that you have the funds to support yourself and your family
  after you arrive in Canada
\end{itemize}

\hypertarget{when-you-arrive-in-canada-youll-meet-an-officer-from-the-canada-border-services-agency-cbsa.-the-officer-will}{%
\subsubsection{When you arrive in Canada, you'll meet an officer from
the Canada Border Services Agency (CBSA). The officer
will:}\label{when-you-arrive-in-canada-youll-meet-an-officer-from-the-canada-border-services-agency-cbsa.-the-officer-will}}

\begin{itemize}
\item
  make sure you're entering Canada before or on the expiry date shown on
  your COPR
\item
  make sure that you are the same person who was approved to travel to
  Canada (they may use your biometrics to do this)
\item
  ask to see your passport and other travel documents
\item
  ask you a few questions to make sure you still meet the terms to
  immigrate to Canada

  \begin{itemize}
  \tightlist
  \item
    the questions will be similar to the ones you answered when you
    applied
  \end{itemize}
\end{itemize}

To help speed up your entry to Canada, keep your passport and other
documents with you at all times. \textbf{Don't pack them in your
luggage}.

\begin{center}\rule{0.5\linewidth}{0.5pt}\end{center}

\hypertarget{disclosure-of-funds}{%
\subsection{Disclosure of funds}\label{disclosure-of-funds}}

\begin{center}\rule{0.5\linewidth}{0.5pt}\end{center}

If you arrive in Canada with \textbf{more than CAN\$10,000}, you must
tell this to the CBSA officer. If you don't tell them, you could be
fined, and your funds could be seized.

What to expect when you land in Canada Knowing what to expect when you
land will contribute to a smooth experience. One of the first people you
will meet at your point of arrival in Canada will be a friendly Canada
customs agent. You will deal with your goods and landing certificates
here. You'll also meet other officials from Immigration Services. They
will ask you to show your passport and visa papers. Immigration
authorities will give you application forms for a variety of documents
that you will need such as your:  Permanent Resident (PR) Card  Social
Insurance Number (SIN) card  Health care card  Driver's license You
will need these cards to find work and start the clock on getting your
provincial healthcare coverage in place. It's important that you start
the process to obtain these documents as soon as you land. When you'll
receive your Permanent Resident (PR) Card Your PR card is proof of your
Canadian permanent resident status. You will need this card whenever you
re-enter Canada. An immigration officer will tell you when you can
expect to receive your PR Card. You will receive your wallet-sized
plastic PR Card in the mail. So, be sure that you receive it within the
time frame indicated by the immigration officer. Important First Steps
to take: 1. Apply for your SIN Card Apply for your Social Insurance
Number (SIN) as soon as you arrive in Canada. Without this number, you
cannot get a job or apply for any government assistance or credit. In
fact, without it, you are virtually a person without an identity in
Canada. 2. Apply for your Health Care Card You will need to apply for a
health care card to receive free medical coverage in the province or
territory where you plan to live. Apply for your health care card as
soon as you land. If you plan to live in British Columbia or Ontario,
there is a three-month waiting period before you will be covered by the
public health care system. So, it's important to apply for your health
care card as soon as you land in either of these provinces. For all
other provinces and territories, you health care coverage begins as soon
as you arrive. Ontario Health Insurance Plan
www.health.gov.on.ca/en/public/programs/ohip Telephone: 1-866-532-3161
3. Find accommodation A big first step is finding accommodation. While
you will have arranged temporary accommodation from your country of
origin, now it's time to find a good place to rent for the medium term.
The best place to search for a rental is the internet and classified
newspapers. You can also visit various neighborhoods, where you will
likely see posters in front of some apartment buildings, advertising
apartments for rent. Please see below links for rental houses and
apartments:  \url{https://housinganywhere.com/s/Toronto--Canada} 
\url{https://www.rentfaster.ca/}  \url{https://rentals.ca/} 
\url{https://www.zumper.com/houses-for-rent/toronto-on} 4. Set up a Bank
Account Another priority in your first few days will be dealing with
your finances. Select a bank or a credit union near your home or work
and open at least one bank account right away. To open an account, you
will need your landed immigrant papers and any other identification you
have to prove your place of residence. Banks provide basic savings
accounts and chequing accounts. You may need a chequing account in
particular because many employers pay by direct deposit right into your
account. Slowly, you can build a good relationship with your bank, and
start building your credit history, by applying for a secured credit
card and then a traditional credit card. To build a strong Canadian
credit history, it's important to pay your bills and loans in a timely
manner. Your credit history will help you when you want to get a
mortgage to buy a home, or get a loan to start a business.
\_\_\_\_\_\_\_\_\_\_\_\_\_\_\_\_\_\_\_\_\_\_\_\_\_\_\_\_\_\_\_\_

\end{document}
